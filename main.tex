\documentclass[12pt,a4paper]{article}

\usepackage[utf8]{inputenc}   
\usepackage{mathtools}
\usepackage[russian]{babel}
\usepackage{graphicx}
\usepackage{geometry}
\geometry{top=2cm, bottom=2cm, left=2cm, right=2cm}
\setcounter{section}{-1}

\title{Продукт: "T-Щит" – Частичная автоматизация оспаривания транзакций в Т-Банке}
\author{Лев Керский}

\begin{document}

\maketitle

\hline

\section{Шаг: Описание компании и задачи}
\textbf{О выбранной компании:} В качестве финтех-компании я выбрал \textbf{Т-Банк}, потому что через него проходит много операций, что важно для моего решения. \\
\textbf{Постановка решаемой проблемы:} Упростить и частично автоматизировать процесс оспаривания транзакций для наших клиентов (владельцев любых карт Т-Банка). Это поможет избежать мошенничества, двойных платежей, проблем с работой банка и др. К тому же, чтобы пользователю оспорить транзакцию, нужны достаточно большие усилия, а мой продукт помог бы упростить им жизнь.
\section{Шаг: Путь клиента (CJM)}
\begin{table}[h]
    \centering
    \resizebox{\textwidth}{!}{ 
        \begin{tabular}{|c|c|}
            \hline
            \textbf{Этап} & \textbf{Текущий CJM} \\
            \hline
            1 & \parbox{0.8\textwidth}{\centering \vspace{2mm}Возникновение проблемы \\ (например, списание за неоказанную услугу)\vspace{2mm}} \\
            \hline
            2 & \parbox{0.5\textwidth}{\centering \vspace{2mm}Поиск информации \\ о способах оспаривания\vspace{2mm}} \\
            \hline
            3 & \parbox{0.5\textwidth}{\centering \vspace{2mm}Подача заявления \\ в банк или долгое общение с тех поддержкой\vspace{2mm}} \\
            \hline
            4 & \parbox{0.5\textwidth}{\centering \vspace{2mm}Сбор и предоставление доказательств \\ (скриншоты, выписки, переписка)\vspace{2mm}} \\
            \hline
            5 & \parbox{0.5\textwidth}{\centering \vspace{2mm}Ожидание решения\vspace{2mm}} \\
            \hline
            6 & \parbox{0.5\textwidth}{\centering \vspace{2mm}Получение результата \\ (успешное оспаривание или отказ)\vspace{2mm}} \\
            \hline
        \end{tabular}
    }
    \caption{Текущий CJM без применения "T-Щит"}
    \label{tab:cjm_current}
\end{table}
Как Вы можете видеть, этот процесс сильно усложнён с точки зрения рядового пользователя, ведь мне просто хочется вернуть свои деньги без лишних проблем. А теперь рассмотрим CJM с добавлением моего решения:
\clearpage
\begin{table}[h]
    \centering
    \centering
    \resizebox{\textwidth}{!}{ 
        \begin{tabular}{|c|c|}
            \hline
            \textbf{Этап} & \textbf{CJM с "T-Щит"} \\
            \hline
            1 & \parbox{0.8\textwidth}{\centering \vspace{2mm}Возникновение проблемы \\ (например, списание за неоказанную услугу)\vspace{2mm}} \\
            \hline
            2 & \parbox{0.5\textwidth}{\centering \vspace{2mm}Запуск "T-Щит" \\ в мобильном приложении банка\vspace{2mm}} \\
            \hline
            3 & \textit{\parbox{0.5\textwidth}{\centering \vspace{2mm}Система автоматически \textbf{*} \\ собирает необходимые данные\vspace{2mm}}} \\
            \hline
            4 & \textit{\parbox{0.5\textwidth}{\centering \vspace{2mm}Система формирует \textbf{*} \\ пакет доказательств\vspace{2mm}}} \\
            \hline
            5 & \textit{\parbox{0.5\textwidth}{\centering \vspace{2mm}Клиент отправляет \\ пакет доказательств в банк одним нажатием (+ мерчанту,если возможно)\textbf{**}\vspace{2mm}}} \\
            \hline
            6 & \parbox{0.5\textwidth}{\centering \vspace{2mm}Ожидание решения\vspace{2mm}} \\
            \hline
            7 & \parbox{0.5\textwidth}{\centering \vspace{2mm}Получение результата \\ (успешное оспаривание или отказ со стороны администраторов)\vspace{2mm}} \\
            \hline
        \end{tabular}
    }
    \caption{CJM с применением "T-Щит"}
    \label{tab:cjm_with_product}
\end{table}
И уже в этом случае клиенту не нужно так упорно собирать все данные о транзакциях, ведь мы можем их "подтянуть"\vspace{} из базы данных. А теперь поговорим об уточнениях:\\

\textbf{*} - Система сама находит по какой-то информации от пользователя данные о совершённой транзакции (например, дата, номер карты, карта получателя и др.), просит его подтвердить свою личность и прислать хотя бы одно доказательство ложной транзакции. Иначе это будет небезопасно для сохранности денег в банке. В эту задачу можно было бы подключить какую-нибудь нейросеть, но только для обработки сообщений с клиентом, требующих небольших человеческих усилий (например,отмена платы за подписку PRO в приложении Т-Банка)\\

\textbf{**} - Так как мы, все-таки, общаемся с двумя сторонами транзакции, то неплохо еще подтвердить возврат средств от мерчанта, чтобы все было честно, но не на все товары можно будет оформить возврат, так как, например, подписку легко отменить, а вернуть условный рюкзак сложнее. Поэтому для начала "Т-Щит" \vspace{} работает только с "простыми" продуктами для возврата, а в будущем мы расширимся и охватим больший рынок возвратов.
\section{Шаг: Добавленная ценность для клиента и банка}
Собственно, в этом и есть вся проблема - мне, как пользователю, не хочется собирать нужные данные для того, чтобы вернуть свои деньги, а мое решение сможет упростить жизнь многим.\\
\textbf{Для клиента:}
\begin{itemize}
    \item Экономия времени и усилий: Автоматизация освобождает клиента от сбора и подготовки документов.
    \item Понижение времени на реагирование о проблеме: Система гарантирует предоставление необходимого пакета информации из базы данных в поддержку Т-Банка.
    \item Более позитивный опыт: Простой и удобный процесс снижает стресс и повышает лояльность к Т-Банку.
    \item Чувство безопасности и защиты: Клиент понимает, что банк заботится о его интересах. \\
\end{itemize}
Так как клиент теперь сильнее заинтересован в нас из-за безопасности, то значит он будет больше покупать и инвестировать в банк. \\
\textbf{Для Т-Банка:}
\begin{itemize}
    \item Снижение операционной нагрузки: Автоматизация уменьшает объём обращений в службу поддержки, так как теперь им приходит только необходимый пакет информации, который не смог обработать ИИ, а не все подряд.
    \item Улучшение клиентского опыта: Лояльные клиенты реже меняют банк.
    \item Увеличение процента успешных оспариваний: Снижение убытков от мошеннических действий.
    \item Увеличение дохода: Более лояльные клиенты активнее пользуются услугами банка.
    \item Укрепление бренда: Использование передовых технологий позитивно влияет на имидж банка.\\
\end{itemize}
Здесь ситуация аналогична, так как если мы укрепляем свой статус на рынке и развиваем передовые отрасли, то в глазах клиента мы более интересны, чем остальные, но \textbf{почему же клиенты готовы этим пользоваться?} Ответ достаточно прост: все хотят в своей жизни комфорта, а "Т-Щит" \vspace{} - еще один повод окунуться в него.
Статьи подтверждающие это:\\
https://futurebanking.ru/post/4052 \\
https://cyberleninka.ru/article/n/metody-otsenki-loyalnosti-klientov-v-roznichnom-segmente-kommercheskogo-banka/viewer (сори за такую некрасивую ссылку)\\
\section{Шаг: Подробное описание продукта "T-Щит"}
Итак, что же по итогу "Т-Щит"? Это сервис, интегрированный в "Т-Среду" (среда всех сервисов Т-Банка) для помощи в возврате денег за какую-либо услугу. Как я уже писал ранее, это не будет отдельным приложением, так как функционала там немного, но тем не менее "Т-Щит" должен быть во всех частях "Т-Среды". \textit{Но что же нам нужно для сбора данных об оспаривании?}
\begin{itemize}
    \item Детали транзакции.
    \item Выписки по карте.
    \item Информацию о клиентах из нашей базы данных.
    \item Историю взаимодействий — данные о предыдущих спорах, возвратах или жалобах на конкретного мерчанта.
    \item Данные от партнеров банка (если это применимо).\\
\end{itemize}
\textbf{"Под капотом Т-Щита":}
\begin{itemize}
    \item API Т-Банка для доступа к данным.
    \item NLP для анализа поступивших запросов.
    \item Алгоритм, который будет проверять задачу на решаемость нейросетью. Если задача оказывается нерешаемой нейросетью, то за дело берется поддержка. \textbf{*}
    \item Микросервисная архитектура для масштабируемости и надежности.
    \item Система динамического обучения модели — нейросеть ежедневно обновляется на основе новых кейсов и решений поддержки.
    \item Интеграция с государственными реестрами мошеннических организаций — автоматическая проверка мерчантов на наличие в "чёрных списках".
\end{itemize}
И в итоге наш сервис будет собирать всю вышеперечисленную информацию, "паковать" \vspace{} 
её в удобный для нейросети/техподдержки вид и отправлять это им, после чего пересылать вердикт пользователю.\\

\textbf{*} - Хочу заметить, что для работы нейросети будут браться "мелкие запросы" \vspace{}, где стоимость возврата меньше, например, 1000 рублей, ведь шанс того, что она ошибется, достаточно большой, а мы не можем этим рисковать.
\section{Шаг: Ожидаемая прибыль}
\begin{itemize}
    \item \textbf{Снижение операционных расходов:} Меньше обращений в службу поддержки, а значит меньше затрат на персонал. $\longrightarrow$ Больше прибыль.
    \item \textbf{Увеличение лояльности клиентов:} Более довольные клиенты сохраняют свои счета. $\longrightarrow$ Расширение клиентской базы. $\longrightarrow$ Больше прибыль.
    \item \textbf{Увеличение транзакций:} Более лояльные клиенты активнее пользуются услугами банка. $\longrightarrow$ Больше охват. $\longrightarrow$ Больше прибыль.
    \item \textbf{Снижение убытков от мошенничества:} Более быстрое и эффективное оспаривание. $\longrightarrow$ Более надежные чем конкуренты. $\longrightarrow$ Больше заинтересованность в Т-Банке. $\longrightarrow$ Расширение клиентской базы. $\longrightarrow$ Больше прибыль.
    \item \textbf{Повышение имиджа банка:} Т-Банк воспринимается как инновационный и \\клиентоориентированный.$\longrightarrow$ Более надежные чем конкуренты. $\longrightarrow$ Больше заинтересованность в Т-Банке. $\longrightarrow$ Расширение клиентской базы. $\longrightarrow$ Больше прибыль.
    \item \textbf{Увеличение конверсии новых клиентов:} Защита транзакций становится конкурентным преимуществом при выборе банка. $\longrightarrow$ Больше охват. $\longrightarrow$ Больше прибыль.
\end{itemize}
Подтверждая свои слова я еще раз попрошу зайти на ссылкам выше, так как основную информацию я брал оттуда.
\end{document}